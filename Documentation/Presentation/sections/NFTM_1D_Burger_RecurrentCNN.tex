\section{NFTM with recurrent CNN}

\begin{secframe}

% CNN Controller of NFTM
\begin{figure}[h]
\centering

\begin{tikzpicture}[
  scale=0.6, every node/.style={transform shape},
  node distance=8mm,
  layer/.style={rectangle,draw,rounded corners,
                minimum width=16mm,minimum height=6mm,align=center}
]

% top row: nu -- field
\node[layer] (nu) {$\nu$};
\node[layer, right=of nu] (field) {$f_t \in \mathbb{R}^{N}$};

% conv stack below field
\node[layer, below=of field] (cat) {concat \\ $(B,2,N)$};
\node[layer, below=of cat] (c1) {Conv1d 2$\to$32 \\ + ReLU};
\node[layer, below=of c1] (c2) {Conv1d 32$\to$64 \\ + ReLU};
\node[layer, below=of c2] (c3) {Conv1d 64$\to$32 \\ + ReLU};
\node[layer, below=of c3] (cout) {Conv1d 32$\to$1};
\node[layer, below=of cout] (delta) {$\Delta f_t$};

% output box to the right, so skip can go right then down
\node[layer, right=25mm of delta] (sum) {$f_{t+1} = f_t + \Delta f_t$};

% connections into conv stack
\draw[->] (field) -- (cat);
\draw[->] (nu.south) |- (cat.west);

\draw[->] (cat) -- (c1);
\draw[->] (c1) -- (c2);
\draw[->] (c2) -- (c3);
\draw[->] (c3) -- (cout);
\draw[->] (cout) -- (delta);

% residual / skip: go right from f_t, then down to sum
\draw[->] (field.east) -- ++(39mm,0) -- ++(0,-10mm) -- (sum.north);
% delta to sum (short arrow)
\draw[->] (delta.east) -- (sum.west);

\end{tikzpicture}
\caption{CNN Controller Architecture.}
\end{figure}
\end{secframe}






% NFTM using Recurrent CNN architecture
\begin{secframe}
\begin{figure}[h]
\centering
% \begin{tikzpicture}[
%     node distance=1.8cm and 1.8cm,
%     box/.style={draw, rounded corners, align=center, minimum width=2.8cm, minimum height=1.1cm},
%     smallbox/.style={draw, rounded corners, align=center, minimum width=2.4cm, minimum height=0.9cm},
%     arrow/.style={->, thick}
% ]

% % True trajectory nodes
% \node[box] (u0) {$u_t^{\text{true}}$};
% \node[box, right=3cm of u0] (uR) {$u_{t+R}^{\text{true}}$};

% % Rollout states
% \node[box, below=2.0cm of u0] (f0) {$f_t$};
% \node[box, right=2.5cm of f0] (f1) {$f_{t+1}$};
% \node[box, right=2.5cm of f1] (f2) {$\dots$};
% \node[box, right=2.5cm of f2] (fRm1) {$f_{t+R-1}$};
% \node[box, right=2.5cm of fRm1] (fRpred) {$f_{t+R}^{\text{pred}}$};

% % Connections from true to rollout start/end
% \draw[arrow] (u0) -- node[left]{\scriptsize init} (f0);
% \draw[arrow] (uR) -- node[right]{\scriptsize target} (fRpred);

% % CNNController blocks between rollout states
% \node[smallbox, above=0.9cm of $(f0)!0.5!(f1)$] (cnn1) {CNNController};
% \node[smallbox, above=0.9cm of $(f1)!0.5!(f2)$] (cnn2) {CNNController};
% \node[smallbox, above=0.9cm of $(f2)!0.5!(fRm1)$] (cnn3) {$\dots$};
% \node[smallbox, above=0.9cm of $(fRm1)!0.5!(fRpred)$] (cnn4) {CNNController};

% % Arrows rollout
% \draw[arrow] (f0) -- (cnn1);
% \draw[arrow] (cnn1) -- node[right]{\scriptsize $f_{t+1}=f_t+\Delta u_t$} (f1);

% \draw[arrow] (f1) -- (cnn2);
% \draw[arrow] (cnn2) -- node[right]{\scriptsize $f_{t+2}=f_{t+1}+\Delta u_{t+1}$} (f2);

% \draw[arrow] (f2) -- (cnn3);
% \draw[arrow] (cnn3) -- (fRm1);

% \draw[arrow] (fRm1) -- (cnn4);
% \draw[arrow] (cnn4) -- node[right]{\scriptsize $f_{t+R}^{\text{pred}}=f_{t+R-1}+\Delta u_{t+R-1}$} (fRpred);

% % Bracket / annotation for no_grad and grad
% \node[align=center, below=0.2cm of f1] (nogradlabel) {\scriptsize $R-1$ pasos \\ \scriptsize sin gradiente};
% \draw[decorate,decoration={brace,amplitude=4pt},yshift=-10pt]
%   ($(f0.south west)+(0,-0.2)$) -- ($(fRm1.south east)+(0,-0.2)$)
%   node[midway,below=6pt]{\scriptsize rollout interno (no\_grad)};

% \node[align=center, below=0.2cm of fRpred] (gradlabel) {\scriptsize 1 paso \\ \scriptsize con gradiente};
% \draw[decorate,decoration={brace,amplitude=4pt},yshift=-10pt]
%   ($(fRm1.south west)+(0,-0.2)$) -- ($(fRpred.south east)+(0,-0.2)$)
%   node[midway,below=6pt]{\scriptsize rollout step (con grad)};

% % Loss box
% \node[box, below=2.2cm of fRpred] (loss) {Loss\\$\mathcal{L}_t = \mathrm{MSE}\big(f_{t+R}^{\text{pred}},\,u_{t+R}^{\text{true}}\big)$};

% \draw[arrow] (fRpred) -- (loss);
% \draw[arrow] (uR) |- (loss);

% \end{tikzpicture}


% \begin{tikzpicture}[
%   scale=0.8, every node/.style={transform shape},
%   node distance=7mm,
%   layer/.style={rectangle,draw,rounded corners,
%                 minimum width=22mm,minimum height=6mm,align=center}
% ]

% % Inputs arriba
% \node[layer] (nu) {$\nu$};
% \node[layer, below=of nu] (ft) {$f_t$};

% % Primer paso CNN
% \node[layer, below=of ft] (cnn1) {CNN};
% \node[layer, below=of cnn1] (f1) {$f_{t+1}$};

% % Segundo paso CNN
% \node[layer, below=of f1] (cnn2) {CNN};
% \node[layer, below=of cnn2] (fk) {$f_{t+K}$};

% % Ground truth y loss
% \node[layer, below=of fk] (truefk) {$f_{t+K}^{\text{true}}$};
% \node[layer, below=of truefk, minimum width=30mm] (loss)
%     {$\mathcal{L} = \|f_{t+K} - f_{t+K}^{\text{true}}\|^2$};

% % Conexiones hacia abajo
% \draw[->] (nu) -- (ft);
% \draw[->] (ft) -- (cnn1);
% \draw[->] (cnn1) -- (f1);
% \draw[->] (f1) -- (cnn2);
% \draw[->] (cnn2) -- (fk);
% \draw[->] (fk) -- (truefk);
% \draw[->] (truefk) -- (loss);

% % Flecha de la loss de vuelta al CNN (siguiente rollout / actualización)
% \draw[->] (loss.east) -- ++(10mm,0) -- ++(0,40mm) -- (cnn1.east);

% \end{tikzpicture}


\begin{tikzpicture}[
  scale=0.5, every node/.style={transform shape},
  node distance=7mm,
  layer/.style={rectangle,draw,rounded corners,
                minimum width=26mm,minimum height=6mm,align=center}
]

% Inputs arriba
\node[layer] (nu) {$\nu$};
\node[layer, right=of nu] (ft) {$f_t$};

% Primer CNN
\node[layer, below=of ft] (cnn1) {CNN};
\node[layer, below=of cnn1] (f1) {$\hat f_{t+1}$};

% Segundo CNN
\node[layer, below=of f1] (cnn2) {CNN};
\node[layer, below=of cnn2] (f2) {$\hat f_{t+2}$};

% Puntos suspensivos
\node[layer, below=of f2, draw=none] (dots) {$\vdots$};

% Último CNN del rollout
\node[layer, below=of dots] (cnnK) {CNN};
\node[layer, below=of cnnK] (fk) {$\hat f_{t+K}$};

% Ground truth y loss
\node[layer, below=of fk] (truefk) {$f_{t+K}^{\text{true}}$};
\node[layer, below=of truefk, minimum width=34mm] (loss)
    {$\mathcal{L} = \|f_{t+K} - f_{t+K}^{\text{true}}\|^2$};

% Conexiones de entrada al primer CNN
\draw[->] (ft) -- (cnn1);
\draw[->] (nu.south) |- (cnn1.west);

% Rollout hacia abajo
\draw[->] (cnn1) -- (f1);
\draw[->] (f1) -- (cnn2);
\draw[->] (cnn2) -- (f2);
\draw[->] (f2) -- (dots);
\draw[->] (dots) -- (cnnK);
\draw[->] (cnnK) -- (fk);
\draw[->] (fk) -- (truefk);
\draw[->] (truefk) -- (loss);

% Flecha de la loss de vuelta al primer CNN (backprop / siguiente rollout)
\draw[->] (loss.east) -- ++(10mm,0) -- ++(0,52mm) -- (cnn1.east);

\end{tikzpicture}

\caption{NFTM with recurrent CNN architecture}
\end{figure}
\end{secframe}