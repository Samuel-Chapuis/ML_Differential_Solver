\section{Hypernetwork 'Brain' Architecture}

% ===== Slide 1: Architecture Diagram (3 IMAGES IN A ROW) =====
\begin{frame}{Hypernetwork 'Brain' Architecture}

% Three images side by side (equal width)
\begin{minipage}[t]{0.32\textwidth}
    \centering
    \includegraphics[width=\textwidth,height=0.75\textheight,keepaspectratio]{images/architecture/Brain-architecture.png}
\end{minipage}%
\hfill
\begin{minipage}[t]{0.32\textwidth}
    \centering
    \includegraphics[width=\textwidth,height=0.75\textheight,keepaspectratio]{images/architecture/Brain-architecture-1.png}
\end{minipage}%
\hfill
\begin{minipage}[t]{0.32\textwidth}
    \centering
    \includegraphics[width=\textwidth,height=0.75\textheight,keepaspectratio]{images/architecture/Brain-architecture-2.png}
\end{minipage}

\end{frame}


% ===== Slide 2: Main Ideas =====
\begin{secframe}
\small
\textcolor{red_unipd}{\Large Main Ideas}

\vspace{0.6em}

\begin{alertblock}{Meta-Learning Framework}
A large neural network (hypernetwork) learns to predict parameters for a smaller CNN that solves the PDE.
\end{alertblock}

\begin{block}{Architecture Components}
\begin{itemize}
  \item \textbf{Features:} Predicts context features from input field
  \item \textbf{Dynamic Attention:} Transformer-style mechanism instead of fixed weights
  \item \textbf{Hypernetwork (Brain):} Predicts equation parameters $\theta$ from context
  \item \textbf{$F_\theta$ (Equation):} Small CNN applies predicted equation to field
\end{itemize}
\end{block}


\end{secframe}