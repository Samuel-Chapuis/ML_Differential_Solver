\section{Introduction and Motivation}

\begin{secframe}
\small
\vspace{-0.1cm}
\begin{block}{Introduction}
3 main neural architectures have emerged to approximate or simulate the behavior of PDEs derived from the \textbf{Navier–Stokes equations}: 
\begin{itemize}
  \item \textbf{PINNs:} include the physical equations directly in their training process so the model learns to follow the known physics (physics-informed training).
  \item \textbf{FNOs:} learn how to map one physical state to the next using patterns in Fourier space.
  \item \textbf{NFTM:} iteratively update predictions through neural “heads” and a controller.
\end{itemize}
\end{block}
\vspace{0.1em}
\begin{block}{Motivation}
The development of NFTMs that combine the accuracy of physics-based learning and the flexibility of data-driven methods to simulate complex PDE systems.
\end{block}
\end{secframe}
