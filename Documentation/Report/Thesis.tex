\documentclass[a4paper,11pt,twoside]{ThesisStyle}

\usepackage{amsmath,amssymb}             % AMS Math
% \usepackage[french]{babel}
%\usepackage[latin1]{inputenc}

\usepackage[utf8]{inputenc}
\usepackage[T1]{fontenc}
\usepackage[left=1.2in,right=1.2in,top=0.8in,bottom=1.1in,includefoot,includehead,headheight=13.6pt]{geometry}
\usepackage{booktabs}
\usepackage{subfig}
\usepackage{xr}


\usepackage{mathrsfs}
\usepackage{booktabs}% http://ctan.org/pkg/booktabs
\newcommand{\tabitem}{~~\llap{\textbullet}~~}
\usepackage{listings,xcolor}
%\usepackage{algpseudocode}
\renewcommand{\baselinestretch}{1.05}
%\renewcommand{\algorithmicforall}{\textbf{for each}}

% Table of contents for each chapter

\usepackage[nottoc, notlof, notlot]{tocbibind}
\usepackage{minitoc}
\setcounter{minitocdepth}{2}
\mtcindent=15pt
% Use \minitoc where to put a table of contents

\usepackage{aecompl}

% Glossary / list of abbreviations

\usepackage[intoc]{nomencl}
\renewcommand{\nomname}{List of Abbreviations}

\makenomenclature

% My pdf code

\usepackage{ifpdf}

\ifpdf
  \usepackage[pdftex]{graphicx}
  \DeclareGraphicsExtensions{.jpg}
  \usepackage[a4paper,pagebackref,hyperindex=true]{hyperref}
\else
  \usepackage{graphicx}
  \DeclareGraphicsExtensions{.ps,.eps}
  \usepackage[a4paper,dvipdfm,pagebackref,hyperindex=true]{hyperref}
\fi
\usepackage{pgfplots}
\graphicspath{{.}{images/}}

% nicer backref links
\renewcommand*{\backref}[1]{}
\renewcommand*{\backrefalt}[4]{%
\ifcase #1 %
(Not cited.)%
\or
(Cited on page~#2.)%
\else
(Cited on pages~#2.)%
\fi}
\renewcommand*{\backrefsep}{, }
\renewcommand*{\backreftwosep}{ and~}
\renewcommand*{\backreflastsep}{ and~}

% Links in pdf
\usepackage{color}
\definecolor{linkcol}{rgb}{0,0,0.4} 
\definecolor{citecol}{rgb}{0.5,0,0} 

% Change this to change the informations included in the pdf file

% See hyperref documentation for information on those parameters

\hypersetup
{
bookmarksopen=true,
pdftitle="...",
pdfauthor="...", 
pdfsubject="...", %subject of the document
%pdftoolbar=false, % toolbar hidden
pdfmenubar=true, %menubar shown
pdfhighlight=/O, %effect of clicking on a link
colorlinks=true, %couleurs sur les liens hypertextes
pdfpagemode=None, %aucun mode de page
pdfpagelayout=SinglePage, %ouverture en simple page
pdffitwindow=true, %pages ouvertes entierement dans toute la fenetre
linkcolor=linkcol, %couleur des liens hypertextes internes
citecolor=citecol, %couleur des liens pour les citations
urlcolor=linkcol %couleur des liens pour les url
}

% definitions.
% -------------------

\setcounter{secnumdepth}{3}
\setcounter{tocdepth}{2}

% Some useful commands and shortcut for maths:  partial derivative and stuff

\newcommand{\pd}[2]{\frac{\partial #1}{\partial #2}}
\def\abs{\operatorname{abs}}
\def\argmax{\operatornamewithlimits{arg\,max}}
\def\argmin{\operatornamewithlimits{arg\,min}}
\def\diag{\operatorname{Diag}}
\newcommand{\eqRef}[1]{(\ref{#1})}

\usepackage{rotating}                    % Sideways of figures & tables
%\usepackage{bibunits}
%\usepackage[sectionbib]{chapterbib}          % Cross-reference package (Natural BiB)
%\usepackage{natbib}                  % Put References at the end of each chapter
                                         % Do not put 'sectionbib' option here.
                                         % Sectionbib option in 'natbib' will do.
\usepackage{fancyhdr}                    % Fancy Header and Footer

% \usepackage{txfonts}                     % Public Times New Roman text & math font
  
%%% Fancy Header %%%%%%%%%%%%%%%%%%%%%%%%%%%%%%%%%%%%%%%%%%%%%%%%%%%%%%%%%%%%%%%%%%
% Fancy Header Style Options

\pagestyle{fancy}                       % Sets fancy header and footer
\fancyfoot{}                            % Delete current footer settings

%\renewcommand{\chaptermark}[1]{         % Lower Case Chapter marker style
%  \markboth{\chaptername\ \thechapter.\ #1}}{}} %

%\renewcommand{\sectionmark}[1]{         % Lower case Section marker style
%  \markright{\thesection.\ #1}}         %

\fancyhead[LE,RO]{}
\fancyhead[RE]{\bfseries\nouppercase{\leftmark}}      % Chapter in the right on even pages
\fancyhead[LO]{\bfseries\nouppercase{\rightmark}}     % Section in the left on odd pages

\renewcommand{\footrulewidth}{0.4pt}
\fancyfoot[LE,LO]{\small Samuel Chapuis, Lucia Victoria Fernandez Sanchez, Alexandra Perruchot-Triboulet Rodriguez}
\fancyfoot[RE,RO]{\bfseries\thepage}

\let\headruleORIG\headrule
\renewcommand{\headrule}{\color{black} \headruleORIG}
\renewcommand{\headrulewidth}{1.0pt}
\usepackage{colortbl}
\arrayrulecolor{black}

\fancypagestyle{plain}{
  \fancyhead{}
  \fancyfoot{}
  \fancyfoot[LE,LO]{\small Samuel Chapuis, Lucia Victoria Fernandez Sanchez, Alexandra Perruchot-Triboulet Rodriguez}
  \fancyfoot[RE,RO]{\bfseries\thepage}
  \renewcommand{\headrulewidth}{0pt}
  \renewcommand{\footrulewidth}{0.4pt}
}

\usepackage{algorithm}
\usepackage[noend]{algorithmic}
%\usepackage[noend]{algpseudocode}
%\usepackage{algpseudocode}

%%% Clear Header %%%%%%%%%%%%%%%%%%%%%%%%%%%%%%%%%%%%%%%%%%%%%%%%%%%%%%%%%%%%%%%%%%
% Clear Header Style on the Last Empty Odd pages
\makeatletter

\def\cleardoublepage{\clearpage\if@twoside \ifodd\c@page\else%
  \hbox{}%
  \thispagestyle{empty}%              % Empty header styles
  \newpage%
  \if@twocolumn\hbox{}\newpage\fi\fi\fi}

\makeatother
 
%%%%%%%%%%%%%%%%%%%%%%%%%%%%%%%%%%%%%%%%%%%%%%%%%%%%%%%%%%%%%%%%%%%%%%%%%%%%%%% 
% Prints your review date and 'Draft Version' (From Josullvn, CS, CMU)
\newcommand{\reviewtimetoday}[2]{\special{!userdict begin
    /bop-hook{gsave 20 710 translate 45 rotate 0.8 setgray
      /Times-Roman findfont 12 scalefont setfont 0 0   moveto (#1) show
      0 -12 moveto (#2) show grestore}def end}}
% You can turn on or off this option.
% \reviewtimetoday{\today}{Draft Version}
%%%%%%%%%%%%%%%%%%%%%%%%%%%%%%%%%%%%%%%%%%%%%%%%%%%%%%%%%%%%%%%%%%%%%%%%%%%%%%% 

\newenvironment{maxime}[1]
{
\vspace*{0cm}
\hfill
\begin{minipage}{0.5\textwidth}%
%\rule[0.5ex]{\textwidth}{0.1mm}\\%
\hrulefill $\:$ {\bf #1}\\
%\vspace*{-0.25cm}
\it 
}%
{%

\hrulefill
\vspace*{0.5cm}%
\end{minipage}
}

\let\minitocORIG\minitoc
\renewcommand{\minitoc}{\minitocORIG \vspace{1.5em}}

\usepackage{multirow}
\usepackage{slashbox}

\newenvironment{bulletList}%
{ \begin{list}%
	{$\bullet$}%
	{\setlength{\labelwidth}{25pt}%
	 \setlength{\leftmargin}{30pt}%
	 \setlength{\itemsep}{\parsep}}}%
{ \end{list} }

\newtheorem{definition}{D�finition}
\renewcommand{\epsilon}{\varepsilon}

% centered page environment

\newenvironment{vcenterpage}
{\newpage\vspace*{\fill}\thispagestyle{empty}\renewcommand{\headrulewidth}{0pt}}
{\vspace*{\fill}}
% POS annotations
\usepackage{ulem}
\usepackage{amssymb}
\usepackage{mathtools}
\usepackage{mathabx}
\usepackage{wasysym}

\newdimen\supsymwidth
\newdimen\supsymheight
\newdimen\tgtsymwidth
\setbox0=\hbox{$_\bigboxvoid$}
\supsymwidth=\wd0
\supsymheight=\ht0
\setbox0=\hbox{$_\bigovoid$}
\tgtsymwidth=\wd0

\newcommand{\supsym}[1]{%
  \mathrlap{\bigboxvoid}
  \raisebox{0.5pt}{\hbox to \supsymwidth{\hfill{\tiny#1}\hfill}}
}

\newcommand{\tgtsym}[1]{
  \mathrlap{\bigovoid}
  \raisebox{0.5pt}{\hbox to \supsymwidth{\hfill{\tiny#1}\hfill}}
}

\newcommand{\enumsentence}[1]{\begin{quotation}#1\end{quotation}}
\newcommand{\evaluator}[1]{\uline{#1}$_{\smiley}$}
\newcommand{\target}[2][\relax]{\uline{#2}$_{\ifx#1\relax\bigodot\else\tgtsym{#1}\fi}$}
\newcommand{\comparator}[1]{#1$_{<}$}
\newcommand{\comparatorthan}[1]{#1$_{>}$}
\newcommand{\superordinate}[2][\relax]{#2$_{\ifx#1\relax\bigovoid\else\supsym{#1}\fi}$}
\newcommand{\attitude}[1]{\uwave{#1}}
\newcommand{\aspect}[1]{\uline{#1}$_{\ast}$}
%3D graphs
\usepackage{tikz-3dplot} 
%Confidential
%\usepackage{draftwatermark}
%\SetWatermarkText{Confidential}
%\SetWatermarkScale{3}
%\SetWatermarkColor[gray]{0.95}

\begin{document}

\pagenumbering{roman}
\begin{titlepage}
\begin{figure}
\vspace{-2cm}
   \begin{minipage}[r]{0.33\linewidth}
   \hspace*{-1.6cm}
      \includegraphics[scale=0.8,  width=.95\textwidth]{images/logocs.jpeg}
   \end{minipage} 
   \begin{minipage}[l]{0.30\linewidth}
   \hspace*{8cm}
      \includegraphics[scale=0.8, width=.95\textwidth]{Images/logops.png}

   \end{minipage}
   
\end{figure}
\vspace*{2cm}

\begin{center}
\noindent \Huge \textbf{Big Data Management \& Analytics Master} \\
\vspace*{0.5cm}
\begin{tabular}{c}
\hline\\
\noindent {\LARGE \textsc{\textbf{Title }}} \\
\noindent {\LARGE \textsc{\textbf{More Lines
}}} \\
\hline\\
\end{tabular}

\vspace*{1cm}
\noindent \LARGE 1st name  \textsc{last name} \\
\vspace*{0.8cm}
Github Link \footnote{ \tiny{
the link to your Github. make sure that the supervisors and me have access. A readme.text must explain how to run your program, using your dataset}}


\end{center}

\begin{center}
\noindent \large 
\begin{tabular}{llcll}
 \textit{Advisor:}	& 1st name \textsc{last name}		& - & \small host institution&  \small email\\
\textit{Advisor:}	& 1st name \textsc{last name}		& - & \small host institution&  \small email

\end{tabular}
\end{center}

\end{titlepage}
\sloppy

\titlepage


\cleardoublepage

%\section*{Introduction}

\tableofcontents

\mainmatter

\chapter{Introduction}
\section{Context and motivation}

\subsection{Project goals}
Our objective is to build a generic neural architecture that can internalize the governing rules of physical systems and then advance their fields over time. The model is trained on challenging PDEs with a focus on fluid mechanics (Navier--Stokes) to stress-test its ability to learn complex, multi-scale, nonlinear dynamics. Beyond matching the training horizon, the same architecture should simulate and extrapolate trajectories past the seen time window, remaining stable and accurate while generalizing across PDE families. Comparing these predictions to reference simulations allows us to identify which ingredients are essential for robustness and transfer.

\subsection{Minimal mathematical frame}
Many PDEs of interest can be written in a (semi-)invariant form:
\begin{equation}
	\partial_t u(x,t) = \mathcal{F}\big(u(x,t), \nabla u(x,t), \nabla^2 u(x,t); \theta\big), \quad x \in \Omega,\; t>0,
\end{equation}
with initial/boundary conditions. Here $u$ is a scalar or vector field; $\mathcal{F}$ encodes advection, diffusion, reaction, constraints; and $\theta$ collects physical parameters (viscosity $\nu$, conductivity $\alpha$, permeability $a(x)$, etc.). Our network seeks to learn a local time-advance operator that approximates $\mathcal{F}$ across multiple PDE families.


\subsection{Historical milestones: AI and PDEs}
\begin{itemize}
	\item \textbf{1990s--2000s}: early universal approximators for simple PDEs; occasional RBF/MLP surrogates.
	\item \textbf{2017--2019}: \textit{Physics-Informed Neural Networks} (PINNs)~\cite{raissi2019physics} enforce PDE residuals in the loss; good on simple geometries, sensitive to stiffness and turbulent regimes.
	\item \textbf{2020}: \textit{Fourier Neural Operator} (FNO)~\cite{li2021fourier} and \textit{Neural Operators}~\cite{kovachki2023neural} learn the operator between function spaces; reference benchmarks on Burgers, incompressible Navier--Stokes, Darcy.
	\item \textbf{2021--2024}: rise of spatio-temporal architectures (Transformers, ConvNeXt) for continuous fields; work on numerical stability, spectral regularization~\cite{rahaman2019spectral}, and conservation of physical quantities.
	\item \textbf{NFTM (2025)}: Neural Field Turing Machine (Malhotra \& Seghouani) \cite{malhotra2025neuralfieldturingmachine} combines \textbf{continuous spatial memory + read/write heads + a neural controller} (CNN/RNN/Transformer). It resembles an explicit scheme: local read, compute an update, local write. Our work builds on this idea to generalize across PDEs.
\end{itemize}

\section{Fluid Mechanics Overview}

\subsection{Why Navier--Stokes?}
	Navier--Stokes models is a canonical set of nonlinear PDEs governing fluid flow, encompassing a wide range of phenomena from laminar to turbulent regimes. Their complexity and multi-scale nature make them an ideal testbed for evaluating the capabilities of neural architectures in learning physical dynamics. Successfully modeling Navier--Stokes would demonstrate the potential of AI-driven approaches in computational fluid dynamics (CFD) and beyond.

\subsection{Governing equations}
The Navier--Stokes equations describe the motion of fluid substances and are derived from fundamental conservation laws: mass, momentum, and energy. They can be expressed as follows:

\begin{align}
	\text{Mass Conervation}:& \quad
    \dfrac{\partial \rho}{\partial t} + \nabla\!\cdot\!(\rho \mathbf{u}) = 0
    \\[16pt]
    \text{Momentum Conervation}:& \quad
    \dfrac{\partial (\rho \mathbf{u})}{\partial t}
    + \nabla\!\cdot\!(\rho \mathbf{u}\otimes\mathbf{u})
    = -\nabla p + \nabla\!\cdot\!\boldsymbol{\Sigma} + \rho \mathbf{g}
    \\[16pt]
    \text{Energy Conervation}:& \quad
    \dfrac{\partial E}{\partial t}
    + \nabla\!\cdot\!((E+p)\mathbf{u})
    = \nabla\!\cdot\!(\boldsymbol{\tau}\!\cdot\!\mathbf{u})
    - \nabla\!\cdot\!\mathbf{q}
    + \rho \mathbf{u}\!\cdot\!\mathbf{g}
\end{align}
\\[8pt]

Where all the variables are defined as:
\begin{itemize}
  \item \(\rho\): fluid density \([\text{kg·m}^{-3}]\)
  \item \(\mathbf{u} = (u,v,w)\): velocity field \([\text{m·s}^{-1}]\)
  \item \(\nabla\!\cdot\!(\rho \mathbf{u})\): divergence of mass flux
  \item \(\partial \rho / \partial t\): local time rate of change of density
  \item \(\rho \mathbf{u}\): momentum density \([\text{kg·m}^{-2}\text{·s}^{-1}]\)
  \item \(\nabla\!\cdot\!(\rho \mathbf{u}\otimes\mathbf{u})\): convective momentum transport
  \item \(-\nabla p\): pressure gradient force per unit volume
  \item \(\boldsymbol{\Sigma}\): viscous stress tensor \([\text{Pa}]\)
  \item \(\rho \mathbf{g}\): body force density (e.g. gravity) \([\text{N·m}^{-3}]\)
  \item \(E = \rho\!\left(e + \tfrac{1}{2}|\mathbf{u}|^2\right)\): total energy density (internal + kinetic)
  \item \(p\): pressure \([\text{Pa}]\)
  \item \(\boldsymbol{\tau}\): stress tensor (viscous + pressure)
  \item \(\mathbf{q}\): heat flux vector \([\text{W·m}^{-2}]\)
  \item \(\rho \mathbf{u}\!\cdot\!\mathbf{g}\): work done by body forces
\end{itemize}

These equations require a viscosity model (e.g., Newtonian fluid) and an equation of state (e.g., ideal gas law or van der Waals equation) to close the system. They form the foundation for simulating fluid dynamics in various applications, from aerodynamics to weather forecasting.

\begin{align}
    \text{Newtonian fluid model}:& \quad
    \begin{cases}
    \boldsymbol{\Sigma} = \mu\left(\nabla \mathbf{u} + (\nabla \mathbf{u})^T\right) + \lambda (\nabla\!\cdot\!\mathbf{u}) \mathbf{I},\\[4pt]
    \boldsymbol{\tau} = \boldsymbol{\Sigma} - p \mathbf{I}
    \end{cases}
    \\[16pt]
    \text{Ideal gas}:& \quad
    p = \rho R T
    \\[16pt]
    \text{van der Waals}:& \quad
    \left(p + a\left(\dfrac{n}{V}\right)^2\right)(V - nb) = nRT
\end{align}

\subsection{Burger's equation as a starting point}
As the project starts, we focus the goal was to use a model simple enough to be learned with limited data and computational resources, yet rich enough to exhibit complex dynamics. Thus, we begin with the burger s equation, a simplified 1D version of the Navier--Stokes equations that captures some features of fluid dynamics, such as shock formation and nonlinear advection.
The viscous Burger's equation in 1D is given by:
\begin{equation}
    \dfrac{\partial u}{\partial t} + u \dfrac{\partial u}{\partial x} = \nu \dfrac{\partial^2 u}{\partial x^2}, \quad x \in [0,L],\; t>0,
\end{equation}

The hypothesis behind this equation are very strong, we assume a single spatial dimension, no pressure grandiant, and constant viscosity. This can be written in a non-dimensional form as:

\begin{align*}
    \text{1D spatial domain}:& \quad
    \begin{cases}
    \mathbf{u} = u(x,t),\\[4pt]
    x \in [0,L]
    \end{cases}
    \\[16pt]
    \text{No pressure gradient}:& \quad
    \nabla p=\mathbf{0}
\end{align*}

\chapter{Related Work}
\label{chap:related_work}
%It is expected to find:

%\begin{itemize}
%\item The research articles  related to your defined problem/objective 
%\item Try to present them by category, the main ideas behind and their limitations
%\item Justify your directions/choices. This part should make a link with the next chapter.

%\end{itemize}

%==============================================================================
\section{Memory Architectures}
\label{sec:related_memory}
%==============================================================================

The foundation of Neural Field Turing Machine approach lies in architectures that augment neural networks with external memory and sophisticated attention mechanisms.

%------------------------------------------------------------------------------
\subsection{Neural Turing Machines}
\label{subsec:ntm}
%------------------------------------------------------------------------------

Graves, Wayne, and Danihelka~\cite{graves2014neural} introduced Neural Turing Machines (NTMs) as differentiable analogues of classical Turing machines. The key innovation is coupling a neural network controller with an external memory matrix that can be read from and written to via learned attention mechanisms, all while remaining end-to-end differentiable~\cite{graves2014neural}.

\paragraph{Architecture.}
An NTM~\cite{graves2014neural} consists of:
\begin{itemize}
    \item \textbf{Controller}: A recurrent or feedforward network that processes inputs and emits control signals
    \item \textbf{Memory matrix}: \(\mathbf{M}_t \in \mathbb{R}^{N \times M}\) with \(N\) addressable locations of size \(M\)
    \item \textbf{Read/Write heads}: Attention-based addressing mechanisms
\end{itemize}

%The controller interacts with memory through two complementary addressing modes:

%\textit{Content-based addressing} uses similarity to focus on relevant memory locations:
%\begin{equation}
%w_t^c(i) = \frac{\exp\left(\beta_t \cdot K[\mathbf{k}_t, \mathbf{M}_t(i)]\right)}{\sum_j \exp\left(\beta_t \cdot K[\mathbf{k}_t, \mathbf{M}_t(j)]\right)}
%\label{eq:ntm_content_addressing}
%\end{equation}
%where \(K[\mathbf{u}, \mathbf{v}] = \frac{\mathbf{u}^T\mathbf{v}}{\|\mathbf{u}\|\|\mathbf{v}\|}\) is cosine similarity and \(\beta_t\) controls focus sharpness.

%\textit{Location-based addressing} enables algorithmic operations via shifting and sharpening:
%\begin{align}
%\tilde{w}_t &= s_t \circledast w_t^g \quad \text{(convolutional shift)} \label{eq:ntm_shift} \\
%w_t(i) &= \frac{[\tilde{w}_t(i)]^{\gamma_t}}{\sum_j [\tilde{w}_t(j)]^{\gamma_t}} \quad \text{(sharpening)} \label{eq:ntm_sharpen}
%\end{align}

While \textbf{Reading}~\cite{graves2014neural} extracts information via weighted average:
\begin{equation}
\mathbf{r}_t = \sum_{i=1}^N w_t^r(i) \mathbf{M}_t(i)
\label{eq:ntm_read}
\end{equation}

\textbf{Writing}~\cite{graves2014neural} combines selective erasure and addition:
\begin{equation}
\mathbf{M}_t(i) = \mathbf{M}_{t-1}(i) \odot [\mathbf{1} - w_t^w(i)\mathbf{e}_t] + w_t^w(i)\mathbf{a}_t
\label{eq:ntm_write}
\end{equation}

%\paragraph{Limitations for PDEs.}
%While groundbreaking for discrete algorithmic tasks, NTMs face fundamental obstacles for PDE solving:
%\begin{enumerate}
%    \item \textbf{Discrete memory structure}: The tabular format \(\mathbf{M} \in \mathbb{R}^{N \times M}\) is designed for discrete symbols, not continuous spatial fields
%    \item \textbf{No geometric structure}: Memory locations lack inherent spatial relationships critical for PDEs
%    \item \textbf{Scalability}: Content-based addressing costs \(\mathcal{O}(N \times M)\) per timestep
%    \item \textbf{Lack of spatial inductive bias}: No assumptions about locality or translation invariance
%\end{enumerate}

\paragraph{Limitations for PDEs.}
While groundbreaking for discrete algorithmic tasks~\cite{graves2014neural}, Neural Turing Machines face fundamental obstacles for Partial Differential Equation 
solving due to their discrete memory structure, which uses a tabular format \(\mathbf{M} \in \mathbb{R}^{N \times M}\) designed for discrete symbols rather than continuous spatial fields. 
Furthermore, their memory locations possess no inherent geometric or spatial relationships, which are critical for representing PDE domains. This design also leads to scalability issues, as content-based addressing incurs a cost of \(\mathcal{O}(N \times M)\) per timestep~\cite{graves2014neural}. Ultimately, the architecture lacks the necessary spatial inductive biases—such as assumptions about locality or translation invariance—that are essential for efficiently learning and solving PDEs.

\paragraph{Relevance.}
NTMs establish the foundational principle we coded, an external memory (the spatial field \(u(x,t)\)) combined with learned read/write operations (our controller implementation). However, we fundamentally adapt this paradigm from discrete memory slots to continuous spatial fields, replacing content-based addressing with local spatial convolutions.

%==============================================================================
\section{Physics-Informed Neural Networks}
\label{sec:related_pinns}
%==============================================================================

Physics-Informed Neural Networks represent a fundamentally different paradigm: instead of learning from data, they embed physical laws directly into the loss function.

%------------------------------------------------------------------------------
\subsection{PINN Framework}
\label{subsec:pinn_framework}
%------------------------------------------------------------------------------

Raissi, Perdikaris, and Karniadakis~\cite{raissi2019physics} introduced PINNs as a method for solving forward and inverse PDE problems by incorporating the governing equations as soft constraints during training.

For the Burgers equation~\cite{raissi2019physics}:
\begin{equation}
\frac{\partial u}{\partial t} + u\frac{\partial u}{\partial x} = \nu\frac{\partial^2 u}{\partial x^2}, \quad (x,t) \in \Omega \times [0,T]
\label{eq:burgers_pinn_pde}
\end{equation}

A PINN approximates \(u(x,t) \approx u_{\theta}(x,t)\) using a deep neural network and minimizes~\cite{raissi2019physics}:
\begin{equation}
\mathcal{L}_{\text{PINN}} = \lambda_f \mathcal{L}_f + \lambda_u \mathcal{L}_u + \lambda_b \mathcal{L}_b
\label{eq:pinn_total_loss}
\end{equation}

%where:
%\begin{align}
%\mathcal{L}_f &= \frac{1}{N_f}\sum_{i=1}^{N_f} \left|f_{\theta}(x_i^f, t_i^f)\right|^2 \label{eq:pinn_physics_loss} \\
%f_{\theta} &= \frac{\partial u_{\theta}}{\partial t} + u_{\theta}\frac{\partial u_{\theta}}{\partial x} - \nu\frac{\partial^2 u_{\theta}}{\partial x^2} \label{eq:pinn_residual} \\
%\mathcal{L}_u &= \frac{1}{N_u}\sum_{j=1}^{N_u} \left|u_{\theta}(x_j^u, t_j^u) - u_j\right|^2 \label{eq:pinn_data_loss} \\
%\mathcal{L}_b &= \frac{1}{N_b}\sum_{k=1}^{N_b} \left|u_{\theta}(x_k^b, t_k^b) - u_k^b\right|^2 \label{eq:pinn_boundary_loss}
%\end{align}

\paragraph{Advantages.}
The approach~\cite{raissi2019physics} is fundamentally \textbf{mesh-free}, eliminating the need for spatial discretization of the domain. This framework is naturally suited for \textbf{inverse problems}, as it can infer unknown physical parameters, such as the viscosity coefficient \(\nu\), directly from sparse and noisy observational data. Furthermore, it provides a \textbf{continuous solution} across space and time, allowing for evaluation at arbitrary query points \((x, t)\). Finally, it is effective in a \textbf{low data regime}, capable of producing accurate solutions with few or even no traditional high-fidelity training measurements~\cite{raissi2019physics}.

\paragraph{Limitations for Burgers Equation.}
Despite their theoretical elegance, Physics-Informed Neural Networks (PINNs) face severe challenges for convection-dominated PDEs such as Burgers' equation. A primary issue is the \textbf{spectral bias toward low frequencies}, where neural networks, as demonstrated by Rahaman et al.~\cite{rahaman2019spectral}, preferentially learn low-frequency components of a solution. For shock-forming solutions rich in high-frequency content, this inherent bias manifests as an inability to accurately represent sharp gradients, leads to oscillatory artifacts (Gibbs phenomenon) near discontinuities, and results in extremely slow convergence, often requiring over \(10^5\) training iterations~\cite{rahaman2019spectral}. This limitation is compounded by a significant \textbf{computational cost}. Training a PINN requires a dense set of collocation points (\(N_f \sim 10^4\) for 1D problems), the computation of high-order derivatives via expensive automatic differentiation, and a lengthy optimization process typically spanning \(50{,}000\) to \(200{,}000\) iterations for convergence~\cite{raissi2019physics}. These factors together present substantial practical barriers for solving convection-dominated flows.

\paragraph{Relevance to Our Work.}
Rather than minimizing PDE residuals, we learn from pre-computed high-fidelity trajectories. However, we retain physics-informed \textit{regularization} without requiring it as the primary loss.

%------------------------------------------------------------------------------

\section{Transformer-based Temporal Modeling}

This section explains the role and functioning of Transformer-inspired models used in this work for temporal prediction tasks.

\subsection{Motivation}

Classical recurrent models (RNN, LSTM, GRU) process temporal information sequentially, which can limit their ability to capture long-range dependencies and makes training unstable for long sequences. Transformers address this limitation by replacing recurrence with an attention mechanism, allowing the model to directly weight the importance of each past time step when making a prediction.

In the context of physical systems such as the Burgers equation, this property is particularly relevant: the future state may depend more strongly on specific past configurations (e.g. shocks or gradients) than on the immediately preceding state alone.

\subsection{General Principle of Transformers}

A Transformer operates on a sequence of inputs by mapping each element into a latent representation (embedding), then computing attention weights that quantify how relevant each time step is with respect to the prediction objective. Formally, attention can be interpreted as a learned weighted average over time, where the weights are data-dependent and normalized using a softmax function.

Unlike full self-attention Transformers used in NLP, the architecture employed here focuses on temporal attention only, which is sufficient for short-to-medium temporal windows and significantly reduces computational cost.

\subsection{TransformerController Architecture}

The \texttt{TransformerController} implemented in this work takes as input a temporal sequence of spatial patches and a physical parameter (the viscosity $\nu$). Each time step is processed independently by a shared encoder, then aggregated using attention.

\paragraph{Input encoding}
At each time step, the spatial patch and the viscosity are concatenated and passed through a multi-layer perceptron to produce a hidden embedding. This step lifts the raw physical variables into a higher-dimensional latent space where nonlinear relationships can be represented.

\paragraph{Temporal attention}
For a sequence of length $L$, the model computes a scalar attention score for each time step. These scores are normalized across time using a softmax function, producing attention weights that sum to one. The final temporal context vector is obtained as a weighted sum of the embeddings:
\[
\mathbf{c} = \sum_{t=1}^{L} \alpha_t , \mathbf{h}_t
\]
where $\alpha_t$ denotes the learned attention weight and $\mathbf{h}_t$ the embedding at time $t$.

\paragraph{Prediction head}
The aggregated context vector is then passed through a fully connected prediction head to output a scalar value. In this application, this scalar corresponds to the predicted value of the field at the center of the spatial patch at the next time step.

\subsection{Interpretability and Advantages}

A key advantage of this Transformer-based controller is interpretability. The attention weights provide direct insight into which past time steps are most influential for a given prediction. This is particularly useful in a physical setting, where one may want to verify whether the model focuses on physically meaningful events (e.g. shock formation).

Compared to recurrent alternatives, this architecture:
\begin{itemize}
\item avoids vanishing or exploding gradients caused by long recursions,
\item enables parallel processing over the temporal dimension,
\item and provides explicit, learnable temporal importance weights.
\end{itemize}

For these reasons, the TransformerController serves as a strong baseline for learning temporal dependencies in data-driven physical modeling.

%
\chapter{Background}
\label{chap:intro}
The objective here is to detail the main concepts/definitions existing algorithms needed to understand your work to be detailed later and to introduce the notations to be used.

Use examples 

Don't forget to cite again these existing approaches



\usetikzlibrary{positioning}
\chapter{Approaches for Neural PDE Simulators}
\label{chap:approaches}

In this chapter, we present two distinct Neural Field Turing Machine (NFTM) model architectures, developed as data-driven simulators for the 1D Burger's equation. Currently, these are defined for the simulation of Burger's dynamics and both approaches utilize the dataset described in Table~\ref{tab:dataset_Burger_statistics}, which consits of 724 training trajectories across 13 different viscosity values (from 0.001 to 0.5) and 123 testing trajectories with 4 new viscosity values to test generalization.\\
The first model \textbf{........ EXPLAIN TRANSFORMER MODEL ........}.\\
The second approach, corresponds to a \textbf{Causal Temporal Convolutional Attention Network (TCAN)} that combines attention with convolutional feature extraction, by restricting attention to past timesteps only and using 1D convolutions for spatial processing.\\
Both approaches enforce physics-informed constraints. We detail their architectures, training procedures, and comprehensive evaluation metrics, highlighting respective strengths and limitations.

% TABLE OF BURGER'S DATASET STATISTICS
\begin{table}[htbp]
\centering
\caption{Burger's Equation Dataset: Number of Trajectories by Viscosity.}
\label{tab:dataset_Burger_statistics}
\resizebox{0.75\textwidth}{!}{
\begin{tabular}{c c c c}
\toprule
\multicolumn{4}{c}{\textbf{Training (724 samples)}} \\
\midrule
\textbf{Viscosity $\nu$} & \textbf{Count} & \textbf{Percentage} & \\
\midrule
0.0010 & 60 & 8.3\% & \\
0.0020 & 40 & 5.5\% & \\
0.0065 & 40 & 5.5\% & \\
0.0080 & 40 & 5.5\% & \\
0.0400 & 40 & 5.5\% & \\
0.0500 & 60 & 8.3\% & \\
0.0700 & 60 & 8.3\% & \\
0.1000 & 60 & 8.3\% & \\
0.1500 & 60 & 8.3\% & \\
0.2500 & 60 & 8.3\% & \\
0.3500 & 60 & 8.3\% & \\
0.4000 & 60 & 8.3\% & \\
0.5000 & 84 & 11.6\% & \\
\midrule
\multicolumn{4}{c}{\textbf{Testing (123 samples)}} \\
\midrule
0.0100 & 50 & 40.7\% & \\
0.0269 & 13 & 10.6\% & \\
0.2000 & 30 & 24.4\% & \\
0.3000 & 30 & 24.4\% & \\
\bottomrule
\end{tabular}
}
\end{table}

% ----------------------------------------------------------- SAMUEL MODEL -------------------------------------------------------------------------------
\section{Approach 1: Transformer based Model}
\subsection{Model Architecture}
\subsection{}


% ----------------------------------------------------------- AKASH MODEL -------------------------------------------------------------------------------
\section{Approach 2: TCAN based Model}
\subsection{Model Architecture}
The Temporal Convolutional Attention Network (TCAN) is a feed-forward autoregressive model that predicts next state of a PDE given a sliding window of past states.
The TCAN serves as the \textit{controller} in the Neural Field Turing Machine model framework, where it learns a discrete-time update rule for continuous spatiotemporal fields.\\
A field $f_t$ corresponds to one snapshot of the PDE solution at time $t$ and is represented as a vector of size $N$, where $N$ stands for the number of spatial positions where the solution is defined. Therefore, the continuous field at time $t$ is given by: $$f_t = [u(x_1, t), u(x_2, t),..., u(x_N,t)] \in \mathbb{R}^{1 \times N}.$$
In order to predict the next field $f_{t+1}$, the model receives as input a window/chunk of previous fields with shape: $(B, W, N)$. This is defined as: $$\text{window} = [f_{t - W + 1}, f_{t - W + 2}, ..., f_{t}],$$ where $B$ is the batch size, $W$ is the history length (number of previous fields in the window), and $N$ is the number of spatial points.\\
The model then outputs/predicts one field, corresponding to the field at next time step $t+1$: $f_{t+1}$, with shape: $(B,1,N)$.\\\\
Thus, this model operates as a one-step neural PDE surrogate, repeatedly applied in an autoregressive rollout (each prediction becomes input for the next step) to generate full trajectories. Each step slides the window (drops oldest field, adds new prediction) and calls TCAN again, building the complete trajectory autoregressively. Figure~\ref{fig:autoregressive-rollout} illustrates this process.
\vspace{0.3cm}
\begin{figure}[htbp]
\centering
\resizebox{0.5\textwidth}{!}{%
\begin{tikzpicture}[
    box/.style={draw, rectangle, rounded corners, minimum width=2.2cm, minimum height=0.8cm, align=center, fill=orange!20, font=\small},
    pred/.style={draw, rectangle, rounded corners, minimum width=1.6cm, minimum height=0.8cm, align=center, fill=green!20, font=\small},
    arrow/.style={->, thick, >=stealth},
    node distance=1.4cm
]

% Step 1
\node[box] (w1) {Window\\$[f_1,\dots,f_{20}]$};
\node[pred, right=of w1] (p21) {$\hat{f}_{21}$};
\node[font=\scriptsize, above=0.10cm of $(w1)!0.52!(p21)$] {TCAN};
\draw[arrow] (w1.east) -- (p21.west);

% Step 2
\node[box, below=of w1] (w2) {Window\\$[f_2,\dots,f_{20},f_{21}]$};
\node[pred, right=of w2] (p22) {$\hat{f}_{22}$};
\node[font=\scriptsize, above=-0.55cm of $(w2)!0.55!(p22)$] {TCAN};
\draw[arrow] (w2.east) -- (p22.west);

% Step 3
\node[box, below=of w2] (w3) {Window\\$[f_3,\dots,f_{21},f_{22}]$};
\node[pred, right=of w3] (p23) {$\hat{f}_{23}$};
\node[font=\scriptsize, above=-0.45cm of $(w3)!0.55!(p23)$] {TCAN};
\draw[arrow] (w3.east) -- (p23.west);

% Pred → next window (curvas por debajo, lejos del texto TCAN)
\draw[arrow] (p21.south) to[out=-60,in=20] (w2.north);
\draw[arrow] (p22.south) to[out=-60,in=20] (w3.north);


% Continuation dots
\node[font=\large] (dots) at ($(w3.south)!0.5!(p23.south) + (0,-1.2)$) {$\dots$};

\end{tikzpicture}%
}
\vspace{0.4cm}
\caption{Autoregressive rollout process: each TCAN call maps a window of previous fields to one predicted field. The prediction is inserted into the next window, replacing the oldest field, to generate the full trajectory. In this example, the model first predicts $f_{21}$ from the window $[f_1,\dots,f_{20}]$, then uses the updated window $[f_2,\dots,f_{20},f_{21}]$ to predict $f_{22}$, and so on.}
\label{fig:autoregressive-rollout}
\end{figure}

\noindent The architecture of this TCAN model consists of \textbf{three main components}:\\
\textbf{1.} A temporal attention encoder that aggregates information across the history window.\\
\textbf{2.} A convolutional decoder that produces a bounded correction.\\
\textbf{3.} A residual update that adds the correction to the last frame.\\\\
The full data flow for a single prediction step is illustrated in Figure~\ref{fig:tcn-flow}.

\begin{figure}[htbp]
\centering
\resizebox{0.95\textwidth}{!}{%
\begin{tikzpicture}[
    box/.style={rectangle, draw, rounded corners, minimum width=2.8cm, minimum height=1cm, align=center, fill=blue!10},
    op/.style={rectangle, draw, rounded corners, minimum width=3cm, minimum height=1cm, align=center, fill=green!10},
    arrow/.style={->, thick, >=stealth},
    input/.style={rectangle, draw, rounded corners, minimum width=3.2cm, minimum height=1.2cm, align=center, fill=orange!20},
    output/.style={rectangle, draw, rounded corners, minimum width=3cm, minimum height=1cm, align=center, fill=red!20},
    node distance=2.0cm
]

% Fila superior: flujo principal
\node[input] (input) {INPUT\\$f_{\text{history}}$};
\node[op, right=of input] (embed) {Embedding \\ Conv1d + GELU\\(frame wise)};
\node[box, right=of embed] (features) {Features\\$(B, W, C, N)$};
\node[box, right=3.2cm of features] (context) {Context\\$(B, C, N)$};
\node[op, right=of context] (decoder) {Decoder\\Conv1d stack};
\node[box, right=of decoder] (corr) {Correction\\$\tanh(\cdot)\times 0.1$};
\node[output, right=of corr] (output) {$f_{\text{next}}$\\$(B, 1, N)$};

% Fila inferior: atención temporal
\node[box, below=2.0cm of features] (query) {Last frame $\rightarrow$ Query $Q$};
\node[box, right=of query] (kv) {All frames $\rightarrow$ Keys $K$, Values $V$};
\node[op, right=of kv] (attn) {Attention\\$\mathrm{softmax}\!\big(\frac{QK^{T}}{\sqrt{C}}\big)$};

% Flechas flujo principal (superior)
\draw[arrow] (input) -- (embed);
\draw[arrow] (embed) -- (features);
\draw[arrow] (context) -- (decoder);
\draw[arrow] (decoder) -- (corr);
\draw[arrow] (corr) -- (output);

% Flecha a Query: igual que antes
\draw[arrow] (features.south) |- (query.north);

% Flecha a Keys/Values: menos baja, gira antes a la derecha
\draw[arrow] (features.east) to[out=-45,in=100] (kv.north);

% Flechas dentro de la rama de atención
\draw[arrow] (query) -- (kv);
\draw[arrow] (kv) -- (attn);

% De atención a contexto: giro bajo
\draw[arrow] (attn.north) to[out=70,in=-80] (context.south);

% Residual de features a context (por arriba)
\draw[arrow] (features.north) to[out=30,in=140] node[above]{residual} (context.north);

% Residual de u_history a u_next (sin texto, menos alto)
\draw[arrow] (input.south) to[out=-45,in=-145] (output.south);

\end{tikzpicture}%
}
\caption{Data flow through the Causal Temporal Attention Network during one prediction step.}
\label{fig:tcn-flow}
\end{figure}

\noindent Figure~\ref{fig:tcn-flow} illustrates the complete data flow through the Causal Temporal Convolutional Attention Network during a single prediction step. Each node performs a specific transformation:

\begin{itemize}
    \item \textbf{$f_{\text{history}}$ (Input)}: Window of $W=20$ previous fields of shape $(B, W, N)$, containing the most recent spatiotemporal snapshots $f_{t-W+1}, \dots, f_t \in \mathbb{R}^N$.
    
    \item \textbf{Conv1d + GELU (frame-wise)}: Applies 1D convolution (kernel size 3) to each of the $W$ fields individually, expanding from 1 to $C=32$ feature channels per field. Output shape: $(B \cdot W, 32, N)$.
    
    \item \textbf{Features $(B, W, C, N)$}: Reshaped embedded representation exposing the temporal dimension, with each of the $W$ fields now containing 32 spatial feature maps.
    
    \item \textbf{Last frame $\to$ Query $Q$}: Extracts features from the most recent field $f_t$ and applies $1\times1$ convolution to generate query vectors $Q \in \mathbb{R}^{B \times 32 \times N}$.
    
    \item \textbf{All frames $\to$ Keys $K$, Values $V$}: Projects all $W$ embedded fields through separate $1\times1$ convolutions to produce keys $K$ and values $V$, both of shape $(B, W, 32, N)$.
    
    \item \textbf{Attention $\mathrm{softmax}\big(\frac{QK^{T}}{\sqrt{C}}\big)$}: Computes causal attention scores between the query (last frame) and all previous frames, producing temporal attention weights $\alpha_{w,p} = \mathrm{softmax}_w(QK^T/\sqrt{32})$ for each spatial position $p$.
    
    \item \textbf{Context $(B, C, N)$}: Aggregates temporal context via weighted sum $\sum_w \alpha_{w,p} V_w$, adds residual connection from last-frame features, and applies GroupNorm. Shape: $(B, 32, N)$.
    
    \item \textbf{Decoder Conv1d stack}: Two-layer convolutional decoder (32$\to$32$\to$1 channels, kernel size 5) that maps rich temporal-spatial context to a scalar correction field. Final layer zero-initialized for training stability.
    
    \item \textbf{Correction $\tanh(\cdot)\times 0.1$}: Applies hyperbolic tangent nonlinearity scaled by 0.1 to bound corrections $|\Delta u| \leq 0.1$, ensuring numerical stability during long rollouts.
    
    \item \textbf{$u_{\text{next}}$ (Output)}: Residual update combining the last input field $u_{\text{history}}[:, -1, :]$ with the predicted correction: $f_{t+1} = f_t + \Delta u$. Shape: $(B, 1, N)$.
\end{itemize}

The two residual connections---from Features to Context (within attention encoder) and from input last frame to output---preserve critical spatiotemporal information while enabling stable incremental updates.




\subsection{Temporal Attention Encoder}

The \texttt{CausalTemporalAttention} module processes the history window through the following steps:

\begin{enumerate}
    \item \textbf{Frame-wise embedding}: Reshape $u_{\text{history}}$ to $(B \cdot W, 1, P)$ and apply a 1D convolution with kernel size 3 followed by GELU activation, producing features of shape $(B \cdot W, C, P)$ where $C=32$ is the embedding dimension.
    
    \item \textbf{Temporal restructuring}: Reshape features back to $(B, W, C, P)$ to expose the temporal dimension.
    
    \item \textbf{Query computation}: Extract features of the last frame $\text{features}[:, -1, :, :] \in \mathbb{R}^{B \times C \times P}$ and apply a $1 \times 1$ convolution to obtain queries $Q \in \mathbb{R}^{B \times C \times P}$.
    
    \item \textbf{Key and value projection}: Apply $1 \times 1$ convolutions to all $W$ frames (flattened to $(B \cdot W, C, P)$ then reshaped) to obtain keys $K$ and values $V$, both of shape $(B, W, C, P)$.
    
    \item \textbf{Causal attention scores}: Compute scaled dot-product attention over the temporal dimension:
    \[
    \text{scores}_{b,w,p} = \frac{\langle Q_b, K_{b,w} \rangle}{\sqrt{C}}, \quad \alpha_{b,w,p} = \text{softmax}_w(\text{scores}_{b,w,p})
    \]
    
    \item \textbf{Context aggregation}: Compute the attended context:
    \[
    \text{context}_b = \sum_{w=1}^W \alpha_{b,w} \cdot V_{b,w} \in \mathbb{R}^{B \times C \times P}
    \]
    
    \item \textbf{Residual projection}: Apply a $1 \times 1$ convolution to the context, add the residual connection to the last frame features, and normalize with GroupNorm.
\end{enumerate}

This mechanism allows the last frame to dynamically query relevant past frames at each spatial location, producing spatially-aware temporal context features.

\subsection{Decoder and Residual Correction}

The \texttt{ImprovedBurgersNet} decoder processes the attention output $\text{context} \in \mathbb{R}^{B \times C \times P}$ as follows:

\begin{enumerate}
    \item \textbf{Convolutional decoder stack}:
    \begin{itemize}
        \item Conv1d: $C \to 32$ channels, kernel size 5, BatchNorm1d, GELU.
        \item Conv1d: $32 \to 1$ channel, kernel size 5, producing $\text{raw\_correction} \in \mathbb{R}^{B \times 1 \times P}$.
    \end{itemize}
    The final layer is zero-initialized to ensure near-zero corrections during early training.
    
    \item \textbf{Bounded correction}: Apply hyperbolic tangent clipping scaled by $\text{corr\_clip}=0.1$:
    \[
    \Delta u = \tanh(\text{raw\_correction}) \cdot 0.1
    \]
    
    \item \textbf{Residual update}: Add the correction to the last input frame:
    \[
    u_{\text{next}} = u_{\text{history}}[:, -1:, :] + \Delta u
    \]
\end{enumerate}

This design enforces incremental updates, preventing instability during long autoregressive rollouts.

\subsection{Autoregressive Training Procedure}

Training employs curriculum learning with multi-step rollouts:

\begin{enumerate}
    \item Initialize $\text{current\_window} = u_{\text{gt}}[:, :W, :]$ from ground truth trajectories.
    
    \item For each rollout step $k \in \{0, \dots, D-1\}$ where $D$ is the rollout depth (curriculum: 8$\to$16$\to$64):
    \begin{itemize}
        \item Predict $\hat{u}_{W+k} = \text{TCAN}(\text{current\_window})$.
        \item Compute composite loss:
        \[
        \mathcal{L} = \text{MSE}(\hat{u}, u_{\text{gt}}) + 0.1 \cdot \|\nabla_x \hat{u} - \nabla_x u_{\text{gt}}\|_2^2 + 0.05 \cdot \mathbb{E}[\max(0, E(\hat{u}) - E(u_t))]
        \]
        where $E(u) = \frac{1}{2} \int u^2 \, dx$ enforces energy dissipation.
        \item Update window: $\text{current\_window} \gets [\text{current\_window}[:, 1:, :], \hat{u}]$ (with occasional teacher forcing).
    \end{itemize}
    
    \item Average loss over rollout, backpropagate through unrolled computation graph, apply gradient clipping, and optimize with AdamW + cosine annealing.
\end{enumerate}

\subsection{Evaluation Metrics}

Full-trajectory predictions are evaluated using:
\begin{itemize}
    \item Standard: MSE, relative $L^2$, PSNR, SSIM, temporal correlation.
    \item Physics-informed: mass conservation error, energy monotonicity fraction, mean PDE residual, spectral error, max gradient error.
\end{itemize}

This comprehensive evaluation ensures both accuracy and physical fidelity across long rollouts.


\chapter{Experiments and Evaluation}
\label{chap:experiments}

\begin{itemize}
\item Objective of Experiments, which measures, which comparisons, evaluations, according to which parameters 
\item Data description
\item Overall program using a figure (API ???) make the link with the components/parts explained in the previous chapter
\item no code
\item Results/interpretation, each table/curve must be explained in the text
\end{itemize}
%
\chapter{Conclusion and Perspectives }
\label{chap:intro}
A summary of your work. More focused on the results

The limitations of the work -> which perspectives/clues to deal with limitations, to improve your work

the last paragraph must be dedicated to the work in team

\section{Gl remarks}
GENERAL :

Each table, figure must be cited and explained in the text.

The references must be complete

Each chapter must start with a paragraph to introduce its content (no need to have a separated for that), except the introduction and the conclusion.
In the same manner each chapter must finish with a paragraph to conclude and to make a link with the next one,  except the introduction and the conclusion.
\section{Gl remarks  about the presentation}

The slides must be numbered

The presentation follows  more and less the structure of the report

No too much blabla about the the gl context you need to define the objectives of the project (with examples if possible) ...

Then how your work  fits into existing works (some main related works), the overall pipeline, your main contributions in this pipeline

also your main contributions in terms of implementation

the main results

Conclusions and next ...

Then the overall 

%\appendix

%\chapter{Appendix}
\label{chap:appendix1}
The progress draft must be included in the appendix


%\include{Appendix2}
%\include{Appendix3}
\nocite{*}  % Include all entries in the bibliography
\bibliographystyle{plain}
\bibliography{Thesis}

%\printnomenclature

\cleardoublepage
\begin{vcenterpage}


\noindent{\large\textbf{Abstract:}}
Solving partial differential equations (PDEs) efficiently remains a fundamental challenge in computational physics, with traditional numerical methods often requiring significant computational resources for complex, multi-scale dynamics. This work presents a Neural Field Turing Machine (NFTM) approach for learning PDE solutions, combining external continuous spatial memory with learned read/write operations through a neural controller. We develop architectures as data-driven simulators for the viscous Burgers equation, a canonical nonlinear PDE that captures essential features of fluid dynamics including shock formation and nonlinear advection. Our approach employs causal temporal attention to aggregate information across history windows, bounded residual corrections for numerical stability during long autoregressive rollouts, and physics-informed loss terms enforcing energy dissipation and gradient accuracy. Trained on 724 trajectories spanning viscosity values from 0.001 to 0.5, the models are evaluated on both standard metrics (MSE, SSIM, PSNR) and physics-informed criteria (mass conservation, energy monotonicity, PDE residual). This framework establishes a foundation for generalizable neural PDE solvers, with planned extensions to two-dimensional Burgers and incompressible Navier-Stokes equations.
\noindent\rule[2pt]{\textwidth}{0.5pt}
{\large\textbf{Keywords:}}
Neural Field Turing Machines, Physics-Informed Neural Networks, Partial Differential Equations, Burgers Equation, Temporal Convolutional Networks, Attention Mechanisms, Autoregressive Models, Scientific Machine Learning, Computational Fluid Dynamics, Deep Learning for Physics.

\end{vcenterpage}

\end{document}
